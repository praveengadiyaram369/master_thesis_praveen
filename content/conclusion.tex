% !TEX root = ../thesis.tex

\chapter{Conclusion}

This master thesis analyzes the challenges involved in retrieving relevant long text documents, such as news articles, for a short keyword query. The proposed approach is evaluated in this study. This pivotal chapter provides a concise summary of the master thesis and highlights a few limitations encountered during the experiment. The chapter is divided into three sections: \textit{Thesis Summary}, \textit{Limitations}, and \textit{Future Work}. The Thesis Summary section compiles essential details from all chapters within the master thesis. The \textit{Limitations} section emphasizes the inherent drawbacks in the proposed approach and underscores potential threats to the validity of our experimental outcomes, ensuring a critical evaluation of the experiment results. The \textit{Future Work} section outlines suggestions to enhance the proposed approach and outlines steps to extend the current research.

\section{Thesis Summary}

The objective of this master thesis was to assess the impact of a candidate keyword selection approach in generating diverse sub-topics during the retrieval of long text documents. The experiment consisted of two stages: evaluating the effectiveness of the candidate keyword selection approach and examining the potential of the selection technique to improve retrieval performance. Additionally, the thesis extends its evaluation through exploratory analysis and literature comparison. The thesis addresses three research questions: RQ1, RQ2, and RQ3. To begin, the thesis introduces the existing work, which is an information retrieval (IR) system initially developed for retrieving news articles at \ac{FKIE}. 

The essential details regarding the \ac{IR} system, such as the process of downloading news articles, their classification, and storage in document indices, are briefly described. The \ac{IR} system employed in this research utilizes lexical and semantic matching algorithms to retrieve relevant news articles based on user search queries. However, the search experience of this system was significantly affected when faced with short queries and lengthy text documents. To address this challenge, a novel approach was proposed in this master thesis to efficiently extract topic representations, specifically subtopics, from the retrieved documents.

After conducting an extensive literature review, a pipeline was developed to carefully select specific keywords that would be instrumental in generating diverse subtopics for a given search query. This selection technique, referred to as \textit{Candidate Keyword Selection}, serves as the central and distinguishing component of the proposed pipeline. Furthermore, an optimized candidate document pool was created to incorporate more representative documents related to the user's query. Essentially, subtopics are derived through term-based clustering of the candidate documents, and experiments were conducted to evaluate the significance of these diverse subtopics on retrieval performance.


RQ1 focuses on evaluating the results of the sub-topic extraction pipeline by employing two target functions on two candidate pools. One of the target functions used is the \textit{Silhouette index}, which assesses the distinctiveness of the generated sub-topics. However, it was observed that this metric failed to capture the diversity of clusters, as it solely relied on the separation between clusters without considering their semantic nature. The second target function, called the \textit{Targeted Negative Document Ratio}, utilizes labeled test set information to determine the distinctiveness of sub-topics. This function performed well when labels were available for all documents in the candidate pool i.e. CP-30. However, due to the lack of labels in the large candidate pool, this evaluation metric proved ineffective. As a result, a manual parameter selection approach was adopted, and the pipeline parameters were finalized based on this process. 

RQ2 expands the evaluation of the sub-topic outcomes and their significance through the use of a survey questionnaire. A dedicated website was developed for survey participants to interact with the sub-topic retrieval system and assess its effectiveness in retrieving relevant news articles, specifically those related to innovation, for a given search query. The feedback from the survey indicated that the generated sub-topics through the proposed approach were distinct from each other. However, the majority of participants expressed that the sub-topics were poorly labeled. Additionally, a quantitative comparison was conducted between two \ac{IR} systems. The first system, \textit{System A}, utilized the sub-topic modeling output, while the other system, \textit{System B}, used a newly generated query through a dynamic template.

The survey results statistically determined that \textit{System A} performed better in retrieving news articles for a given query and sub-topic. However, upon careful analysis of \textit{System A}'s results, it was observed that the poor retrieval outcomes were due to the choice of sub-topic being completely irrelevant to the query. Furthermore, one of the survey questions focused on whether the sub-topics truly helpful to users in finding relevant news articles. The feedback to this question revealed that nearly half of the participants found the retrieval process with sub-topics to be beneficial.

RQ1 and RQ2 focused on evaluating sub-topic generation, while RQ3 assessed the retrieval performance for various combinations of sub-topic ranking and document ranking pairs using the Mean Average Precision (MAP) metric. Eight different \ac{IR} systems, ranging from IR0 to IR7, were proposed and compared against each other. The \ac{IR} system that performed the best was then compared with the existing literature. Among the IR systems, IR4 exhibited the highest \ac{MAP} results. IR4 utilized sub-topic ranking based on the decreasing order of the number of documents within each cluster, combined with document ranking based on decreasing query similarity. The retrieval performance of the IR4 system was compared to literature that employed a \textit{Cross-encoder} for generating ranked results. Both approaches utilized a multi-stage ranking architecture.

At the first stage, three different retrieval approaches, namely \textit{BM-25, Bi-encoder, and Candidate pool}, were used to retrieve candidate documents. In the second stage, the IR4 system approach was compared to the cross-encoder using \ac{MAP} as the evaluation metric. The results of the comparison indicated that the IR4 ranking approach performed the best when using the candidate pool as the retriever. However, the cross-encoder generated better rankings of news articles when candidate documents were retrieved using BM-25 and bi-encoder.

\section{Limitations}

The primary objective of the target users at \ac{FKIE} is to explore news articles related to innovation in the military and technology fields. The candidate keyword selection method proposed in the master thesis relies solely on query similarity to choose keywords that can generate diverse sub-topics. However, one limitation of this approach is that it does not directly address the retrieval of innovation-related documents; instead, it focuses on segmenting the documents into clusters.

Additionally, relying solely on one feature, namely query similarity, for keyword selection is not ideal and may result in inefficient keyword selection. The context-based keyword extraction technique employed in this approach selects the top 25 noun chunks as keywords but ignores their relevance distribution within the news article. The true potential of the proposed approach is explored when all noun chunks are considered as candidate keywords, although this may introduce computational challenges. Nonetheless, selecting specific noun chunks based on their importance in a news article, considering their distribution, could help mitigate this drawback.


The absence of sub-topic labels in the test set has necessitated the selection of pipeline parameters using intrinsic and extrinsic cluster evaluation approaches. However, the chosen target functions for parameter selection have revealed a lack of effectiveness in capturing distinctiveness, leading to the need for manual parameter selection. The limitations of the employed target functions in the master thesis are discussed in Section 7.1.6. Manual parameter selection introduces the risk of human bias and proves highly inefficient, particularly when dealing with a large number of parameters. To optimize sub-topic extraction, the thesis implemented the caching of keyword embeddings, resulting in an average real-time extraction time of 7.75 seconds. Although this optimization has reduced extraction time, it is important to note that compared to current search engines, which often deliver results in just a few seconds, the sub-topic retrieval approach remains relatively slower. This difference in speed may potentially impact user satisfaction negatively.

The survey questionnaire and its evaluation are subject to certain limitations. While the survey participants are randomly selected regardless of their background, they may not fully represent the target users, introducing the possibility of non-representative sample bias. Another limitation arises from the lack of explicit definitions for certain queries or phrases used in the questionnaire, such as "Distinctive," "Well-labeled," "IT-Bedrohungsanalyse," "Unbemannte Landsysteme," and so on. Although these terms may seem self-explanatory, they can still be subject to interpretation by participants who are unfamiliar with these phrases in both German and English.

Additionally, the survey question assessing the usefulness of sub-topic retrieval does not gather system and sub-topic information from the participants. This missing information creates the potential for response bias, and the analysis of this specific question in the survey cannot be fully justified.

\section{Future Work}

To effectively address innovation-related news articles, the distinctive sub-topics extracted through the proposed approach in the master thesis can be further expanded to include sub-topic classification. By tracking user information from the sub-topic retrieval system, it becomes possible to observe which sub-topics are of high interest to users. This differentiation among sub-topic clusters proves valuable in identifying and eliminating non-innovation-related clusters. Instead of solely relying on the similarity between keywords and the search query, the candidate keyword selection technique can incorporate additional information related to technology, military, security, and other relevant domains. Integrating these features enhances the criteria for keyword selection and also improves the output of sub-topic clustering.

In the present research, it has been observed that the silhouette score is not effective in capturing the distinctiveness of the cluster output, despite the clustering demonstrating better separation. Further investigation is required to develop an efficient evaluation metric that can assess the distinctiveness of sub-topics considering semantic information. Additionally, there is room for further optimization of sub-topic generation to reduce the average extraction time from 7.75 seconds to less than 2 seconds. Based on the evaluation from the survey, it has been determined that the sub-topics are not well-labeled. This highlights the need for improvement in the naming of clusters using the cluster centroid technique, or the exploration of more robust techniques for labeling.

This master thesis utilized the \ac{USE} at multiple stages within the proposed approach to encode the query, noun phrases, and news articles. However, since the completion of this research, several more efficient and state-of-the-art sentence encoding models have been published. It is imperative to evaluate the current research using these newly developed models to ensure its performance remains up-to-date.

Additionally, the performance of the multi-stage ranking architecture employed in this research will be evaluated against the recent advancements made in \textit{\ac{LLM}}. Unlike the approach used in this thesis, where relevant documents are extracted for a given query, \ac{LLM}s focus on generating relevant answers directly from the knowledge source. Furthermore, \ac{LLM}s have the potential to be utilized for generating sub-topics, thereby providing an alternative to the traditional topic modeling or clustering approach.


