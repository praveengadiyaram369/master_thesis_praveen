% !TEX root = ../thesis.tex

% Kurzfassung in Deutsch und Englisch
\begin{otherlanguage}{ngerman}
    \section*{Kurzfassung}
 \ac{TR} beinhaltet das Abrufen relevanter Textdokumente aus einer Sammlung als Antwort auf eine Suchanfrage. \ac{TR} ist ein Teilbereich des \ac{IR}, der sich auf textbasierte Datenquellen und abgerufenen Daten konzentriert. Eine der größten Herausforderungen in der IR besteht darin, hochrelevante Dokumente als Top-Ergebnisse für eine bestimmte Suchanfrage zu finden. Diese Herausforderung wird noch schwieriger, wenn der Benutzer eine bestimmte Absicht verfolgt, aber keine kontextbezogenen Informationen in der Suchanfrage enthält. So kann beispielsweise eine Suchanfrage wie \emph{Robotik} Dokumente aus verschiedenen Bereichen wie Herstellung, Landwirtschaft, Militär usw. ergeben. Eine einfache Schlüsselwortsuche kann den Benutzer mit vielen falsch-positiven Ergebnissen überfordern, wenn der Benutzer die Innovationsdokumente nur in Bezug auf einen bestimmten Bereich, z. B. \emph{Militär}, untersuchen möchte. Das Abrufen von Textdokumenten, wie z. B. Nachrichtenartikeln oder längeren Texten, erschwert die Abrufaufgabe aufgrund der Fülle von Wörtern zusätzlich. Eine effiziente Suche hängt in hohem Maße von der Indizierung der Dokumente ab, insbesondere wenn es sich um große Textdokumentensammlungen handelt.

Um diese Herausforderungen zu adressieren, wird durch die Kombination von lexikalischen und semantischen Matching-Techniken ein optimierter Kandidatendokumenten-Pool erstellt. Dieser Ansatz ermöglicht die Generierung eines höchst repräsentativen Dokumentenpools, der für die Benutzeranfrage geeignet ist. Um die abgerufenen Dokumente innerhalb des Pools genau darzustellen, wird in dieser Masterarbeit eine Kandidatenschlüsselauswahltechnik auf Basis der Ähnlichkeit zur Anfrage vorgeschlagen und bewertet. Die generierten Dokumentrepräsentationen werden als \emph{sub-topics} bezeichnet, die unterschiedliche Wege für den Zugriff auf die abgerufenen Dokumente eröffnen. Der vorgeschlagene Ansatz wird mit einem unsupervised neuronalen Re-Ranking-Verfahren aus der Literatur verglichen und zeigt eine bessere Leistung, wenn ein optimierter Kandidatenpool verwendet wird. Zusätzlich wird eine manuelle Bewertung der abgerufenen Ergebnisse mit Hilfe eines Umfragebogens durchgeführt, bei der die Mehrheit der Teilnehmer der vorgeschlagenen Methode zur Erstellung unterschiedlicher Sub-topics zugestimmt hat. Darüber hinaus wird eine explorative Analyse durchgeführt, um die Leistung der auf der Grundlage von Sub-topics-Rankings entwickelten Retrievalsysteme zu bewerten.\\
\\
\\
\\
\\
\\
\\
\\
\\
\\
\\
\\
\\
\\
\\
\\
\\
\\
\\


\end{otherlanguage}

\begin{otherlanguage}{english}
    \section*{Abstract}

Text Retrieval (TR) involves retrieving relevant text documents from a collection in response to a search query. \ac{TR} is a branch of Information Retrieval (IR) that focuses on text-based data sources and retrieved data. One of the main challenges in \ac{IR} is retrieving highly relevant documents as the top results for a given query. This challenge becomes more difficult when the user has a specific intention but lacks contextual information in the search query. For instance, a user query like \emph{Robotics} can yield documents from multiple domains, such as manufacturing, agriculture, military, etc. A simple keyword search can overwhelm the user with many false positives when the user wants to explore the innovation documents related only to a specific domain, such as \emph{Military}. Retrieving text documents, such as news articles or lengthy texts, further complicates the retrieval task due to the abundance of words. Efficient retrieval heavily relies on document indexing, particularly when dealing with large text document collections.

To address these challenges, an optimized candidate document pool is created through the combination of lexical and semantic matching techniques. This approach ensures the generation of a highly representative document pool suitable to the user query. To accurately represent the retrieved documents within the pool, a candidate keyword selection technique based on query similarity is proposed and evaluated in this master thesis. The generated document representations are referred to as \emph{sub-topics,} which establish distinct pathways for accessing retrieved documents. The proposed approach is compared to a unsupervised neural re-ranking technique from the literature and demonstrates better performance when an optimized candidate pool is used. Additionally, a manual evaluation of the retrieved results is conducted using a survey questionnaire, where the majority of participants have agreed that the proposed methodology for creating distinguishable sub-topics. Moreover, exploratory analysis is conducted to assess the performance of retrieval systems developed based on sub-topic rankings.


\end{otherlanguage}
